\documentclass[11pt]{letter}
\usepackage[a4paper,left=2.5cm, right=2.5cm, top=1cm, bottom=1cm]{geometry}
\usepackage{hyperref}
\usepackage[osf]{mathpazo}
\signature{Thomas Guillerme \\ (on behalf of my co-authors)}
\address{The University of Queensland \\School of Biological Sciences \\St Lucia QLD 4067, Australia \\guillert@tcd.ie}
\longindentation=0pt
\begin{document}

\begin{letter}{}
\opening{Dear Editors,}

Using patterns in multidimensional spaces to study biological processes is now a common toolkit in ecology and evolution.
In such analysis, researchers typically use matrices where traits or transformed-traits are columns (e.g. anatomical measurements, community compositions) and observations are rows (e.g. specimens, field sites, etc.).
These matrices are commonly referred to as morphospaces in evolution or trait-space in ecology.
It is then possible to look at how observations or groups of observations occupy the trait space to understand biological processes.
For example, if plant community A occupies more space trait-space than community B, the former is could be more diversed;
or if the morphospace of a group of organisms increases after colonisation of an island, it could be experiencing an adaptive radiation.
However, surprisingly very little work has been done on characterising what should be measured in these multidimensional analysis: what should be measured in a trait-space to accurately capture the pattern of difference between two communities or how trait-space occupancy changes through time?
Furthermore, the parallel between these analysis in ecology and evolution has never been clearly acknowledged (to our knowledge).

In our research article, entitled ``Shifting spaces: how to summarise multidimensional spaces occupancy?'', we provide the first interdisciplinary review of 25 space occupancy metrics that uses the broad classification of metrics into volume, density and position to capture pattern changes in trait space.
We assess the behaviour of metrics using simulations and a selection of interdisciplinary empirical datasets; these cover a wide range of potential data types and evolutionary/ecological questions.
We also introduce a tool for Measuring Occupancy in Multidimensional Space ([`moms`](@@@)), which is a user-friendly open-source graphical interface tool to allow the tailored testing of metric behaviour for any use case.
This will allow researchers to comprehensively assess the properties of their trait-space and metrics associated with their specific biological question.

Furthermore, we are are convinced that open data and reproducible papers are the key part of the future of academia.
Therefore, this entire paper and its supplementary is easily reproducible and based on open access dataset.
In fact, the paper is wrapped in a \texttt{R} package format and can be compile as a vignette through \url{https://github.com/TGuillerme/moms}.
We believe that this extra care put into making the paper easy to reproduce will foster not only a better understanding of multidimensional analysis but also future analysis on how the finding of this paper related beyond the fields of ecology and evolution.

We look forward to hearing from you soon,

\closing{Yours sincerely,}


\end{letter}
\end{document}
